\documentclass{dcbl/challenge}

\setdoctitle{Using Libraries}
\setdocauthor{Stephan Bökelmann}
\setdocemail{sboekelmann@ep1.rub.de}
\setdocinstitute{AG Physik der Hadronen und Kerne}
\usepackage{listings}
\usepackage{hyperref}

\begin{document}


\section*{Aufgaben}
\begin{aufgabe}
    Research the following C-standard library functions on this website: \href{https://en.cppreference.com/w/c/header}{cppreference.com}
    Write a program, that uses the following functions:
    \begin{enumerate}
        \item \texttt{double pow(double, double)} from \texttt{math.h}
        \item \texttt{double sqrt(double)} from \texttt{math.h}
        \item \texttt{double sin(double)} from \texttt{math.h}
        \item \texttt{int rand(void)} from \texttt{stdlib.h}
        \item \texttt{int putchar(int)} from \texttt{stdio.h}
        \item \texttt{int getchar(void)} from \texttt{stdio.h}
        \item \texttt{int printf(const char *, ...)} from \texttt{stdio.h}
        \item \texttt{int scanf(const char *, ...)} from \texttt{stdio.h}
        \item \texttt{time\_t time(time\_t *arg)} from \texttt{time.h}
        \item \texttt{double difftime(time\_t, time\_t)} from \texttt{time.h}
    \end{enumerate}
    For which of these did you need to link the libraries manually?
\end{aufgabe}

\begin{aufgabe}
    Now install a non-standard library.
    Let's take \texttt{libjansson} for example. 
    It can be installed with \texttt{sudo apt install libjansson-dev}.
    We can use libjansson to serialize and deserialize structs to JSON and vice versa.
    Use the following C-code:
    \begin{lstlisting}
    #include <jansson.h>
    #include <stdio.h>
    
    typedef struct {
        int id;
        char name[20];
    } person_t;
    
    int main(void) {
        person_t person = {1, "John Doe"};
    
        // new JSON-dataobject
        json_t *root = json_object();
        
        // add data
        json_object_set_new(root, "id", json_integer(person.id));
        json_object_set_new(root, "name", json_string(person.name));
    
        // convert to string
        char *json_string = json_dumps(root, JSON_ENCODE_ANY);
    
        // write to stdout
        printf("%s\n", json_string);
    
        // cleanup
        free(json_string);
        json_decref(root);
    
        return 0;
    }
    \end{lstlisting}  
    How do you need to link the library for it to compile?      
\end{aufgabe}

\begin{aufgabe}
    Finally, we'd like to use an open source project and compile it ourselves. 
    We start by cloning the project \texttt{curl}, which is a library for transferring data over the internet.
    \begin{lstlisting}
        git clone https://github.com/curl/curl.git
    \end{lstlisting}
    Since this is a complex project, a simple \texttt{makefile} doesn't do the trick anylonger, but we will actually have to run a script to generate a \texttt{makefile} for it.
    For this we need to install some dependencies:
    \begin{lstlisting}
        sudo apt install libssl-dev build-essential autoconf libtool -y
    \end{lstlisting}
    This command also installs the \texttt{openssl}-library.
    Let's build the project:
    \begin{lstlisting}
        cd curl
        ./buildconf
        ./configure --with-openssl
        make
    \end{lstlisting}
    Now we can install the library:
    \begin{lstlisting}
        sudo make install
    \end{lstlisting}

    To use the library, we'd like to write a program, that takes a \texttt{URL}, downloads its contents and writes it to a file.
    Use the following code for the program: \href{https://gist.github.com/bjoekeldude/af6b9ff61d55a550f31aa22eb3e596ca}{Clone this Gist}.
    You can now compile the program with only one line: What does this line look like?
    Run the program with the following command: \texttt{./<name of your program> <URL> <file>}.
\end{aufgabe}


\section*{Annotations}
\begin{enumerate}
    \item Introduction to make and autotools: \url{https://www.youtube.com/watch?v=WFLvcMiG38w\&pp=ygUJYXV0b3Rvb2xz}
\end{enumerate}

\end{document}
